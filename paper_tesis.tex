%\usemodule[ntb-to-xtb]
\showframe

\setupbodyfont[termes, 12pt]

\setuppagenumbering[
	state=start,
	location={lower,middle}
	]

\setuplayout[
	backspace=2.5cm,
	margin=1.5cm,
	marginedge=0.5cm,
	header=1cm,
	footer=1cm,
	width=16cm
	]

\setupcolumns[n=2]

\starttext
\title{Article}

\subject{Abstract} %%%%%%%%%%%%%%%%%%%%%%%%%%%%%ABSTRACT%%%%%%%%%%%%%%%%%%%%%%%%%%%

Tabacay micro-basin is very important for 27 communes that lives in it, from which they extract resources, develop economic activities and life overall. This micro-basin is also important for Azogues city, which takes water from its rivers for supply the population. This study analyzes land use dynamics and vegetation’s quality during 2005-2017 period, using Landsat satellite images along with geographic information systems, for determining critical areas for management and conservation of Tabacay micro-basin. Four land use classes were determined: agricultural, forestry, urban and moor. Analysis results show that agricultural and urban land use has grown in surface terms, as well as forestry and moor surface has decreased. Moor land use has been the most affected by land use dynamics. In terms of surface, this land use shows a 6 ha/year loss during analysis period, which is equivalent to 54\% of its initial surface. Analysis also shows that most affected zone by forest and moor fragmentation and surface loss is micro-basin’s high part, having as main cause of the problem the agricultural frontier expansion and urbanization.\blank

Keywords: Geographic information systems, Landsat, land use, Tabacay, micro-basin, multitemporal analysis, remote sensing.

\subject{Introduction}%%%%%%%%%%%%%%%%%%%%%%%%%%%%%INTRODUCTION%%%%%%%%%%%%%%%%%%%%%%%%%%%

Tabacay micro-basin has a critical role in Azogues city development. Also for 27 communes which lives in its territory. Although its vital role, there are various social and environmental issues affecting, overall, medium and upper part.\par

Two of this issues are erosion and landslides, which together are destroying potential farmlands, increasing sediment disposal in many streams. Soil erosion is caused, mainly, for deforestation and inappropiate cropping practices.\par%(EMAPAL EP, 2005)

Another issue affecting Tabacay micro-basin is urban growth to upper part, being this consequence of building an massive, unplanned and poor thecnical road network.\par%(EMAPAL EP, 2005)

Absense of information about LULC evolution is a constraint for Tabacay micro-basin management. Studies about this variables allows understanding of causes and consequences of deforestation, degradation, desertification and biodiversity loss processes in a specific area.%(Mas, Velásquez y Counturier, 2009) 
LULC assesments are also crucial for rural growth planning and food safety monitoring.\par%(Sahle, Marohn, & Cadisch, 2015)

\subject{Materials and methods}%%%%%%%%%%%%%%%%%%%%%%%%%%%%%MATERIALS AND METHODS%%%%%%%%%%%%%%%%%%%%%%%%%%%

\subsubject{Study area}
Tabacay micro-basin is located in Cañar province, Ecuador. It also belongs to Paute basin.%See fig. X
The area covers a surface of 68.3 Km\high{2}. Its altitude varies from 2490 m to 3730 m above sea level. 

\subsubject{Land use classes}
Four land use classes were defined. Specifically: agricultural, forestry, urban and moorland. 

\subsubject{Landsat imagery}
Three Landsat images were used. The details are described in \in{Table}[t1].
\par


\setupcaption[table][location=top, width=\textwidth, align=middle]
\setupTABLE[r][each][align=middle, frame=off] 
\setupTABLE[r][first][style=\bf, topframe=on, bottomframe=on]
\setupTABLE[r][last][bottomframe=on]

\placetable[][t1]{Detail of satellital imagery}
{\bTABLE
\bTR \bTD Image date \eTD \bTD Sensor \eTD \bTD Resolution \eTD \eTR
\bTR \bTD 2004, dec 29 \eTD \bTD ETM+ \eTD \bTD 30 m \eTD \eTR
\bTR \bTD 2016, nov 20 \eTD \bTD OLI \eTD \bTD 30M \eTD \eTR
\eTABLE}


Atmospheric and topographic correction were made for allowing comparison between images.


\subject{Results and discuss}%%%%%%%%%%%%%%%%%%%%%%%%%%%%%RESULTS AND DISCUSS%%%%%%%%%%%%%%%%%%%%%%%%%%%

\setupcaption[table][location=top, width=\textwidth, align=middle]
\setupTABLE[r][each][align=middle, frame=off]
\setupTABLE[r][1][style=\bf, bottomframe=on]
\setupTABLE[r][2][style=\bf, bottomframe=on, align={middle,lohi}]
\setupTABLE[c][1][style=\bf]
\setupTABLE[c][2][style=\bf]
\setupTABLE[r][8][bottomframe=on]

\placetable[][t1]{Transition matrix for study period}
{\bTABLE
%1st ROW
\bTR \bTD[nc=2] \eTD \bTD[nc=4,align={middle,lohi}] 2017 \eTD \bTD \eTD \bTD \eTD \eTR
%2nd ROW
\bTR \bTD  \eTD \bTD \eTD \bTD Agricultural \eTD \bTD Forestry \eTD \bTD Urban \eTD \bTD  Moorland \eTD \bTD Total\\2005 \eTD \bTD Loss \eTD \eTR
%3rd ROW
\bTR \bTD[nr=4,align={middle,lohi},bottomframe=on] 2005 \eTD \bTD Agricultural \eTD \bTD[style=\bf] 3738,5 \eTD \bTD 197,4 \eTD \bTD 143,6 \eTD \bTD 2,6 \eTD \bTD 4082,2 \eTD \bTD 343,6 \eTD \eTR
%4th ROW
\bTR \bTD Forestry \eTD \bTD 564,2 \eTD \bTD[style=\bf] 1026,1 \eTD \bTD 20,2 \eTD \bTD 6,7 \eTD \bTD 1617,3 \eTD \bTD 591.1 \eTD \eTR
%5th ROW
\bTR \bTD Urban \eTD \bTD 28,6 \eTD \bTD 0,4 \eTD \bTD[style=\bf] 664,5 \eTD \bTD 0,0 \eTD \bTD 693,5 \eTD \bTD 29,0 \eTD \eTR
%6th ROW
\bTR[bottomframe=on] \bTD Moorland \eTD \bTD 166,2 \eTD \bTD 67,7 \eTD \bTD 0,0 \eTD \bTD[style=\bf] 203,8 \eTD \bTD 437,7 \eTD \bTD 233.8 \eTD \eTR
%7th ROW
\bTR \bTD[nr=2,bottomframe=on] \eTD \bTD Total 2017 \eTD \bTD 4497,5 \eTD \bTD 1291,7 \eTD \bTD 828,3 \eTD \bTD 213,1 \eTD \bTD[style=\bf] 6830,6 \eTD \bTD \eTD \eTR
%8th ROW
\bTR \bTD Gain \eTD \bTD 759,0 \eTD \bTD 265,5 \eTD \bTD 163,8 \eTD \bTD 9,3 \eTD \bTD \eTD \bTD \eTD \eTR
\eTABLE}


\subject{Conclusions}%%%%%%%%%%%%%%%%%%%%%%%%%%%%%CONCLUSIONS%%%%%%%%%%%%%%%%%%%%%%%%%%%




\subject{References}%%%%%%%%%%%%%%%%%%%%%%%%%%%%%REFERENCES%%%%%%%%%%%%%%%%%%%%%%%%%%%